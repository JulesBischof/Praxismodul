\documentclass[]{article}

% Pakete

\usepackage[german]{babel} % Sprache: Deutsch
\usepackage{graphics} % grafik package
\usepackage{amsmath} % mathematische Symbole
\usepackage{hyperref} % hyperlinks im Dokument
\usepackage[utf8]{inputenc} % UTF-8 Formatierung
\usepackage[T1]{fontenc} % Umlaute

\usepackage{parskip} % Absätze anders Formatieren
\usepackage[table]{xcolor} % Paket für Farben in Tabellen

% Titel, Autor und Datum
\title{Dokumentation PRAX\_6}
\author{Julian Bischof \\
        Hochschule Luzern \\
        julian.bischof@stud.hslu.ch}
\date{\today}

% beginn des Dokuments:
\begin{document}

% Titelseite

\maketitle 
\thispagestyle{empty}

\newpage
\thispagestyle{empty}
\begin{abstract}
    Abstract hier
\end{abstract}
\newpage

% Inhaltsverzeichnis
\pagenumbering{roman}
\tableofcontents % Inhaltsverzeichnis
\listoffigures % Abbildungsverzeichnis -> alles was sich innerhal von figure umgebung befindet begin/end{figure}, caption{text}
\listoftables % Tabellenverzeichnis -> alles was sich innerhalb von table umgebung befindet
\clearpage
\setcounter{page}{0}
\pagenumbering{arabic}

\section{Einleitung}

Einleitung hier

\section{Hintergründe}

Etwas zu Randbedingungen

\section{Methodik und Vorgehen}
In einem ersten Schritt wurde die Aufgabenstellung in verschiedene Bausteine unterteilt, 
die im Rahmen eines morphologischen Kastens zusammengesetzt wurden. Um die verschiedenen Konzepte
auch im Rahmen der Herstellungskosten miteinander vergleichen zu können, wurden die verschiedenen Bausteine
auch preislich bewertet. Diese Vorkalkulationen beziehen sich auf Preise und Daten, welche entweder vom Hersteller zur
Verfügung gestellt werden, oder von Onlinehändlern -- z.B. von Mouser.ch -- bezogen werden können.

Im folgenden wird kurz auf einzelne Bausteine eingegangen, bevor die verschiednen Varianten kurz beschrieben werden. 

\subsection{Morphologischer Kasten}

\begin{itemize}
    \item Chip:
        \\ Hier wurden verschiedene Treiberbausteine Betrachtet. Sowohl vollintegrierte Treiber, welche
        sowohl einen Indexer, einen Rampgenerator mitsamt Endstufe besitzen als auch einfache Indexer mit
        Endstufen. Aus diesem Punkt folgen zwangsläufig auch I\textsubscript{rms} und U\textsubscript{out}. 
    \item Anzahl Erweiterungen: 
        \\ Dieser Punkt stellt die Erweiterbarkeit der verschiedenen Konzepte dar. Maximal gewünscht sind
        nur 2.
    \item Schnittstelle zu IOe2:
        \\ Die betrifft die Schnittstelle, über welche das Board Konfigueritert und angesteuert werden kann.
    \item Adresskonfiguration, SPI Erweiterungen
        \\ Die eindeutige Identifikation der Erweiterung stellt ein Problem dar, welches es zu lösen gibt. 
        Als Möglichkeiten wurde eine einfache inkrementelle verbindung der Schnittstellen, ein binäres "herunterzählen" 
        über Logikbausteine als auch eine Adresskonfiguration über Shiftregister als möglich gezeigt. 
    \item 
\end{itemize}




\section{Theoretische Grundlagen}

Etwas zu meinen Recherchen
\subsection{I2C Bus}
\subsection{SPI Bus}

\section{Konzepterstellung}
\section{Randbedingungen}
\section{Komponentenauswahl}
\section{Analyse und Entscheid}

\section{Schemaerstellung}

\section{Zwischenstand}

\section{Fazit und Ausblick}

%\bibliography{reference}

\section{Anhang}

\end{document}
