\documentclass[]{article}

% Pakete

\usepackage[german]{babel} % Sprache: Deutsch
\usepackage{graphics} % grafik package
\usepackage{amsmath} % mathematische Symbole
\usepackage{hyperref} % hyperlinks im Dokument
\usepackage[utf8]{inputenc} % UTF-8 Formatierung
\usepackage[T1]{fontenc} % Umlaute

\usepackage{csquotes}
\usepackage{biblatex}
\addbibresource{04_references.bib}

\usepackage{parskip} % Absätze anders Formatieren
\usepackage[table]{xcolor} % Paket für Farben in Tabellen

% Titel, Autor und Datum
\title{Dokumentation PRAX\_6}
\author{Julian Bischof \\
        Hochschule Luzern \\
        julian.bischof@stud.hslu.ch}
\date{\today}

% beginn des Dokuments:
\begin{document}

% Titelseite

\maketitle 
\thispagestyle{empty}

\newpage
\thispagestyle{empty}
\begin{abstract}
    Abstract hier
\end{abstract}
\newpage

% Inhaltsverzeichnis
\pagenumbering{roman}
\tableofcontents % Inhaltsverzeichnis
\listoffigures % Abbildungsverzeichnis -> alles was sich innerhal von figure umgebung befindet begin/end{figure}, caption{text}
\listoftables % Tabellenverzeichnis -> alles was sich innerhalb von table umgebung befindet
\clearpage
\setcounter{page}{0}
\pagenumbering{arabic}

\section{Einleitung}

Einleitung hier

\section{Hintergründe}

Etwas zu Randbedingungen

\section{Theoretische Grundlagen}
\subsection{Methodik}
Etwas zu meinen Recherchen
\subsection{I2C Bus}
\subsection{SPI Bus}

\section{Konzepterstellung}
Anhand eines Moprhologischen Kastens sind für die verschiedenen Teilprobleme verschiedene Umsetzungsmöglichkeiten
zusammengesetzt worden. Um die verschiedenen Konzepte auch angesichts der Herstellungskosten miteinander vergleichen und abwägen 
zu können, wurden die Kosten der einzelnen Blöcke bewertet. Diese Vorkalkulationen beziehen sich auf Preise und Daten, welche entweder vom 
Hersteller zur Verfügung gestellt werden, oder von Onlinehändlern -- z.B. von Mouser.ch -- bezogen werden können.

\subsection{Morphologischer Kasten}

Im folgenden wird kurz auf einzelne Bausteine eingegangen, bevor die verschiednen Konzepte kurz beschrieben werden:

\begin{itemize}
    \item Chip:
        \\ Hier wurden verschiedene Treiberbausteine betrachtet. Sowohl vollintegrierte Treiber, welche
        sowohl einen Indexer, einen Rampgenerator mitsamt Endstufe besitzen als auch einfache Indexer mit
        Endstufen. Aus diesem Punkt folgen zwangsläufig sowohl I\textsubscript{rms}, U\textsubscript{out}, als auch ob
        eine externe Endstufe benötigt wird. 
    \item Anzahl Erweiterungen: 
        \\ Dieser Punkt stellt die Erweiterbarkeit der verschiedenen Konzepte dar. Maximal gewünscht sind
        nur 2. 
    \item Schnittstelle zu IOe2:
        \\ Die betrifft die Schnittstelle, über welche das Board Konfigueritert und angesteuert werden kann. Zur auswahl stehen hierbei 
        Konfigurationen über SPI-Bus, I2C-Bus, UART oder eine Indexer-Schnittstelle, bei welcher das IOe Schritte und Richtung vorgeben muss.
    \item Adresskonfiguration, SPI Erweitern
        \\ Umgesetzt werden könnte dies durch inkrementelles Verbinden der Schnittstille von der Eingangs- zur Ausgangsschnittstelle,
        Oder durch spezielle logikverknüpfungen, welche entsprechende Chipselect oder Adressleitungen auf die richtigen Pins setzen. 
    \item Anschluss von Sensoren
        \\ Es ist möglich Sensoren wie z.B. Referenzschalter an eigene Eingänge auf dem Erweiterungsboard anzuschliessen, über Leiterbahnen an 
        die Schnittstelle weiterzugeben oder sogar eine Elektronik auf dem Board zu haben, welche die Signale direkt verarbeiten könnte. 
    \item Auswertung der Encoder
        \\ Der Encoder ist kein Pflichtkriterium. Trotzdem ist es möglich diesen entweder auf der Erweiterung selbst auszuwerten oder ihn an die
        Schnittstelle weiterzugeben. 
    \item Ramp Generation
        \\ Die Aufgabe der Bandplanung kann vom IOe selbst übernommen werden, oder aber von einer Elektronik, welche sich auf der Erweiterung befindet.
        Dies könnte z.B. eine MCU sein, oder aber ein Integrierter Treiber von z.B. Trinac. 
    \item Spannungsversorgung
        \\ Die Spannungsversorgung könnte bei entsprechender Filterung vom IOe2 bezogen werden oder aber externb mit einem Anschlusskabel gespiesen werden.
\end{itemize}

\subsection{Konzepte}
Im Anhang sind alle Konzepte nochmals aufgellistet. 

Die Konzepte teilen alle die Eigenschaft, dass sie durch die inkrementell verbundenen Leiter Erweiterbar gestaltet wurden. Die Adresskonfiguration durch
Schieberegister oder Logische Verknüpfung zeigt sich als unnötigen Kostenpunkt, auf den aufgrund des geringen Budgets gerne verzichtet wurde. 
Im Wesentlichen haben sich 4 Konzepte zur finalen Bewertung herauskristallisiert:
\begin{itemize}
    \item Konzept 1
    \\ Der vollintegrierte Treiber von Trinamic, TMC5240, kann Motoren mit einer Spannung von 24V bei 2.1A treiben. Konfigueritert wird dieser Treiber
    via SPI-Bus. Dieser Treiber bringt von Haus aus sowohl eine Möglichkeit zur Encoderauswertung als auch zur Sensorenauswertung mit. Das Board wird
    mit einer eigenen 24V Schnittstelle mit Strom versorgt. Die Bandplanung wird in diesem Fall ebenfalls vom Trinamic Chip übernommen.
    \item Konzept 7
    \\ Die TI-Treiber DRV8462 besitzen lediglich einen Indexer als auch die Endstufe integriert im Chip. Ein grosser Vorteil dieser Variante stellt die 
    Chipfamilie von TI dar, bei welcher 3 verschiedene Chips 1:1 austauschbar sind und somit also auch Secondsource Themen in gewisser Weise schon abgehandelt sind.
    Um möglichst nahe an die vorgegebenen 30CHF heranzukommen, beziehen diese Konzepte ihre Speisung direkt über die IOe2 Schnittstelle und verwenden ebenfalls
    die Digitalen Eingänge des IOe2's um Sensoren auszuwerten. Der Encoder muss hierbei über die Schittstelle an das IOe2 weitergereicht werden. 
    \item Konzept 8
    \\ Dieses Konzept ist eine Mischen aus Konzept 1 \& 7. Hier wurde wieder der ADI-Trinamic Chip TMC5240 eingesetzt, welcher weiter oben schon beschrieben wurde.
    Um mit dem Preis günstiger zu werden ist besitzt dieses Konzept keine eigene Speisung mehr, sondern wird ebenfalls über die Schnittstelle zum IOe2 gespiesen. 
    Die Digitalen Eingänge werden ebenfalls vom IOe2 bezogen. Ein grosser Vorteil dieses Chips ist die Möglichkeit, Encoder auswerten zu können. Aus diesem Grund 
    ist diese Schnittstelle vorhanden geblieben. 
\end{itemize}

\subsection{Bewertung und Entscheid}
Die Konzepte sind anhand einer Nutzwertanalyse gegeneinander verglichen und vorbewertet worden. Da im Pflichtenheft die Herstellungskosten als ein besonders
hohes und wichtiges Kriterium herausgehoben wurden, sind diese Punkte nochmals höher bewertet als alle anderen Kriterien. 

Im Übrigen sind die Kriterien nach dem Schema Basiskriterium = 4, X1 = 3, X2 = 2 und X3 = 1 gewichtet. Bei dieser Bewertung sind die oben genannten 3 Konzepte als 
am besten bewertet worden. In einer grösseren Runde, gemeinsam mit den Projektbetreuern, ist die Entscheidung für das Konzept Nr. 7 gefallen und wird im weiteren
ausgearbeitet.  

\section{Randbedingungen}

Da das Erweiterungsboard die Speisung aus von der Schnittstelle zum IOe2 bezieht, muss sichergestellt werden, dass sich dort auch genug Pins befinden um 
den geforderten Strom zu führen. Laut Datenblatt können pro Pin mindestens 1.7A geführt werden \cite[siehe][S. 11]{Datasheet_Interface}

\section{Komponentenauswahl und Dimensionierungen}
\section{Analyse und Entscheid}

\section{Schemaerstellung}

\section{Zwischenstand}

\section{Fazit und Ausblick}

\printbibliography

\section{Anhang}

\end{document}
