\documentclass[]{article}

% Pakete

\usepackage[german]{babel} % Sprache: Deutsch
\usepackage{graphics} % grafik package
\usepackage{amsmath} % mathematische Symbole
\usepackage{hyperref} % hyperlinks im Dokument
\usepackage[utf8]{inputenc} % UTF-8 Formatierung
\usepackage[T1]{fontenc} % Umlaute


% Titel, Autor und Datum
\title{Dokumentation PRAX\textunderscore6}
\author{Julian Bischof \\
        Hochschule Luzern \\
        julian.bischof@stud.hslu.ch}
\date{\today}

% beginn des Dokuments:
\begin{document}

% Titelseite

\maketitle 
\thispagestyle{empty}

\newpage
\thispagestyle{empty}
\begin{abstract}
    Abstract hier
\end{abstract}
\newpage

% Inhaltsverzeichnis
\pagenumbering{roman}
\tableofcontents % Inhaltsverzeichnis
\listoffigures % Abbildungsverzeichnis -> alles was sich innerhal von figure umgebung befindet begin/end{figure}, caption{text}
\listoftables % Tabellenverzeichnis -> alles was sich innerhalb von table umgebung befindet
\clearpage
\setcounter{page}{0}
\pagenumbering{arabic}

\section{Einleitung}

Einleitung hier

\section{Hintergründe}

Etwas zu Randbedingungen

\section{Methodik und Vorgehen}

Etwas zu Methodik und Vorgehen

\section{Theoretische Grundlagen}

Etwas zu meinen Recherchen
\subsection{I2C Bus}
\subsection{SPI Bus}

\section{Konzepterstellung}
\section{Randbedingungen}
\section{Komponentenauswahl}
\section{Analyse und Entscheid}

\section{Schemaerstellung}

\section{Zwischenstand}

\section{Fazit und Ausblick}

%\bibliography{reference}

\section{Anhang}

\end{document}
